\documentclass{beamer}
\usepackage[danish]{babel}
\usepackage{pdfpages}
\usepackage{graphicx} 
\usepackage{hyperref}
\usepackage{pgfpages}
\setbeameroption{show notes on second screen}
\usepackage{minted}
\usepackage{comment}

\usetheme[progressbar=frametitle]{metropolis}
\usepackage[sfdefault]{FiraSans} %% option 'sfdefault' activates Fira Sans as the default text font
\usepackage[T1]{fontenc}
\renewcommand*\oldstylenums[1]{{\firaoldstyle #1}}



%   Vi ser gerne, at jeres præsentation inkluderer:
%   - Et overblik over evt. afvigelser, begrænsninger, og/eller tilføjelser ift.    
%     kravsspecifikationen.
%   - En kort demo.
%   - Et overblik over jeres arkitektur. Der er ikke tid til at forklare alle dele, så %     fremhæv de dele, I finder mest relevante.
%   - Refleksioner over jeres design og projektet som helhed.


\title{IMADA fagråd}
\institute{Syddansk Universitet - Institut for Matematik og Datalogi}
\date{16. September 2025}

\begin{document}

\frame{\titlepage}

\begin{frame}{SDU}
\begin{figure}
    \includegraphics[width=1\textwidth]{SDU.png}
\end{figure}
\note{
\begin{itemize}
    \item Over 19.000 studerende på tværs af 5 fakulteter.
\end{itemize}
}
\end{frame}

\begin{frame}{NAT}
\begin{figure}
    \includegraphics[width=1\textwidth]{NAT.png}
\end{figure}
\note{
\begin{itemize}
    \item Naturvidenskab er et af de 5 fakulteter.
    \item Herunder tilhører Biomolekylær Biologi, IMADA, Fysik-kemi-farmaci og Biologisk Institut.
    \item Institut for Matematik og Datalogi, som vi kalder IMADA, er altså en del af naturvidenskab.
    \item Sammen med de andre institutter på naturvidenskab, har vi et godt fællesskab.
\end{itemize}
}
\end{frame}

\begin{frame}{IMADA}
\begin{figure}
    \includegraphics[width=0.7\textwidth]{fagrådbillede.png}
\end{figure}
\note{
\begin{itemize}
    \item På IMADA har vi 7 bacheloruddannelser og 7 kandidatuddannelser.
    \item Vi er omkring 1000 studerende på IMADA når man tæller både bacheloruddannelser og kandidater.
\end{itemize}
}
\end{frame}

\begin{frame}{Fagråd}
\begin{figure}
    \includegraphics[width=0.7\textwidth]{fagrådbillede.png}
\end{figure}
\note{
\begin{itemize}
    \item På IMADA har vi Fagrådet, som repræsenterer de studerende på IMADA.
    \item Bindeled mellem studerende og institut.
    \item Fagrådet arbejder for at forbedre studieforholdene.
    \item Desuden står Fagrådet for at skabe et stærkt fællesskab blandt de studerende.
    \item Det gør vi ved at arrangere sociale og faglige arrangementer.
    \item Alle studerende på IMADA er automatisk medlem af fagrådet.
    \item Derfor opfordrer vi alle der har ideer eller lign. til at kontakte os.
\end{itemize}
}
\end{frame}

\begin{frame}{Fagråd}
\begin{figure}
    \includegraphics[width=0.8\textwidth]{rundbold.jpeg}
\end{figure}
\note{
\begin{itemize}
    \item Her ser i et billede fra et af vores sociale arrangementer, nemlig rundbold.
    \item Rundbold er et årligt tilbagevendende arrangement.
\end{itemize}
}
\end{frame}

\begin{frame}{Fagråd}
\begin{figure}
    \includegraphics[width=0.8\textwidth]{årshjul.png}
\end{figure}
\note{
\begin{itemize}
    \item Vi har lavet et årshjul over nogle af de arrangementer vi laver i fagrådet.
    \item Det hænger på opslagstavlen i saunaen, samt ude i gangen.
    \item Nogle arrangementer er tilbagevendende, andre er mere spontane.
    \item Sociale arrangementer som f.eks. Mal din egen kop, skakturneringer, karaokegrillfest, fredagsbar og meget mere.
    \item Faglige arrangementer som f.eks. workshops og virksomhedsbesøg.
    \item Det er vigtigt at i forstår at vi i fagrådet er her for jer. Så hvis i har lyst til at komme med ideer eller ønsker, så tøv ikke med at kontakte os.
\end{itemize}
}
\end{frame}

\begin{frame}{Fagråd}
\begin{figure}
    \includegraphics[width=0.6\textwidth]{fbqr.png}
    \text{Facebook \& kontakt@imadafagraad.dk}
\end{figure}
\note{
\begin{itemize}
    \item vi har en facebookside, hvor vi poster nyheder og arrangementer. 
    \item Vi har også en mail, som vi tjekker jævnligt.
\end{itemize}
}
\end{frame}

\begin{frame}{Fagråd}
\begin{figure}
    \includegraphics[width=0.8\textwidth]{saunaen.png}
\end{figure}
\note{
\begin{itemize}
    \item Vi har et fagrådslokale, som vi kalder saunaen.
    \item Det er et lokale vi selv står for at vedligeholde.
    \item Desuden har vi 2 køleskabe, hvor vi sælger BILIGE øl og sodavand.
    \item Vi har også gratis kaffe og te til alle IMADA studerende. Vi får sponsoreret af vores gode venner fra Foreflight her i Odense.
\end{itemize}
}
\end{frame}

\begin{frame}{Fagråd}
\begin{figure}
    \includegraphics[width=0.5\textwidth]{fagrådPlakat.png}
\end{figure}
\note{
\begin{itemize}
    \item Fagrådet består af (lige nu) 10 bestyrrelsesmedlemmer fra IMADA.
    \item Vi er fordelt på forskellige årgange og studieretninger og har alle forskellige interesser og kompetencer.
    \item Man kan vælges til fagrådet ved at stille op til valg 1 gang om året i marts måned.
    \item Men man kan også valgt ind ved et ekstraordinært valg, der er 10 eller flere medlemmer, der ønsker det.
\end{itemize}
}
\end{frame}

\begin{frame}{Festministeriet}
\begin{figure}
    \includegraphics[width=0.7\textwidth]{imadafestministerie.jpg}
\end{figure}
\note{

\begin{itemize}
    \item Festudvalget står for blandtandet fredagsbaren, 2 årlige fester og andre sociale arrangementer.
    \item Vi holder blandt andet IMADA dato på viggos her på lørdag den 20. samt en fest med FKF (ÆTER) den 10 oktober.
\end{itemize}
}
\end{frame}
\begin{frame}{Festministeriet}
\begin{figure}
    \includegraphics[width=0.5\textwidth]{festplakat.png}
\end{figure}
\note{
\begin{itemize}
    \item Hvis i har lyst til at være med i festudvalget, så holder vi generalforsamling d. 9 oktober. (der kommer plakater op ift.)
    \item Alle er velkomne, og vi opfordrer især nye ansigtet til at komme og høre mere om, hvad vi laver.
\end{itemize}
}
\end{frame}

\end{document}